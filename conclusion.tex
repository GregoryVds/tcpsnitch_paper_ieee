\section{Discussion}\label{sec:conclusion}

We have proposed \tcpsnitchns, an application that intercepts network
system and library calls on the Linux and Android platforms to
collect more information about their usage, including the parameters
passed to those API calls. We collected more than 2.3 millions calls made by
90 popular applications on sixteen thousands sockets. The application and the
collected dataset are publicly available\footnote{The entire dataset can be
explored via \url{https://tcpsnitch.org}. The \tcpsnitch source code is
available from \url{https://github.com/GregoryVds/tcpsnitch} and the web
interface can be retrieved from \url{https://github.com/GregoryVds/tcpsnitch_web}.
}.

Our analysis revealed several interesting patterns for the utilization
of the socket API on Android applications. First, in an IPv6 enabled
WiFi network, these applications prefer IPv6 over IPv4. Second, UDP sockets are
mainly used as a shortcut to retrieve information about the network
configuration. Third, many Android applications use the same pattern
of system calls to establish and terminate TCP connections. Fourth, Android
applications use various socket options, even some like
\texttt{TCP\_INFO} that are not directly exposed by the standard Java
API.

\tcpsnitch and its associated website already provide a good overview
of how real applications use the socket API. Our future work will be
to add traces from more applications in the database and support other
platforms starting with MacOS.
